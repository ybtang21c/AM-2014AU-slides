\begin{frame}{第十八讲、可导函数的单调性与极值}
	\linespread{1.5}
	\begin{enumerate}
	  \item {\bf 内容与要求}{\b( \S 5.4 )}
	  \begin{itemize}
	    \item 熟练掌握可导函数单调性的判定方法
	    \item 熟练掌握可导函数极值的判定方法
% 	    \item 掌握函数的分析作图法
% 	    \item 熟练掌握L'Hospital法则
	  \vspace{1em}
	  \end{itemize}
	  \item {\bf 课后练习:}
	  \begin{itemize}
	    \item 书面作业:\\ {\b 习题5.4:6(2,4),8,9(1),11,13}
 	    \item 思考题:{\b 习题5.4:4,5,12}
	  \end{itemize}
	\end{enumerate}
\end{frame}

\section{可导函数的单调性}

\begin{frame}{可导函数的单调性}
	\linespread{1.2}\pause 
% 	$f'(x)>0\;(a<x<b)\quad \Rightarrow \quad f(x)$在$(a,b)$上严格单调递增
	\begin{block}{{\bf 定理5.4.1}\hfill P263}
		设$f(x)$在$[a,b]$上连续,$(a,b)$内可导,且$f\,'(x)$恒大(小)于零,则$f(x)$在
		$[a,b]$上严格单调递增(减)。
	\end{block} \pause
	\begin{itemize}
	  \item {以上仅仅是判定可导函数严格单调的充分条件,而非充要条件}\pause 
	  \item {若定理中的“大(小)于”改成“大(小)于等于”,则对应于单调递增情形,且为充要条件}
	\end{itemize}
\end{frame}

% \begin{frame}[t]{可导函数的单调性}
% 	\linespread{1.2}\pause 
% % 	$f'(x)>0\;(a<x<b)\quad \Rightarrow \quad f(x)$在$(a,b)$上严格单调递增
% 	\begin{block}{{\bf 定理5.4.1}\hfill P263}
% 		设$f(x)$在$[a,b]$上连续,$(a,b)$内可导,且$f\,'(x)$恒大(小)于零,则$f(x)$在
% 		$[a,b]$上严格单调递增(减)。
% 	\end{block} 
% 	\only<3>{\begin{exampleblock}{{\bf 例1}\hfill}
% 		设$f(x)\in C[a,b]$,在$(a,b)$内$f\,''(x)>0$,
% 		证明:
% 		$$F(x)=\df{f(x)-f(a)}{x-a}$$
% 		在$(a,b)$内严格单调递增。
% 	\end{exampleblock}}
% 	\pause \pause
% 	\begin{itemize}
% 	  \item \alert{以上仅仅是判定可导函数严格单调的充分条件,而非充要条件}\pause 
% 	  \item \alert{若定理中的“大(小)于”改成“大(小)于等于”,则对应于单调递增情形,且为充要条件}
% 	\end{itemize}
% \end{frame}

\begin{frame}
	\linespread{1.2}
	\begin{exampleblock}{{\bf 例1}\hfill P264:例1-2}
		讨论$y=x-\sin x$的单调性。
	\end{exampleblock}\pause 
	\bigskip
	\begin{alertblock}{{\bf 定理}(可导函数单调的充要条件)\hfill}
		设$f(x)$在$[a,b]$上连续,$(a,b)$内可导,则$f(x)$在$[a,b]$上严格单调递增,
		当且仅当:
		\begin{enumerate}
		  \item $f\,'(x)\geq 0,\;x\in(a,b)$
		  \item 在$(a,b)$的任意子区间上$f\,'(x)$不恒为零
		\end{enumerate}
	\end{alertblock}\pause
	{\bf 注:}导函数只有{\bb 孤立零点}的函数严格单调
\end{frame}

\begin{frame}{单调性的应用}
	\linespread{1.2}
	\begin{alertblock}{{\bf 推论1}\hfill}
		设$\varphi(x),\psi(x) $均在$[a,b]$上可导,且:
		\begin{enumerate}
		  \item $\varphi'(x)>\psi'(x),\;x\in(a,b)$
		  \item $\varphi(a)=\psi(a)$
		\end{enumerate}
		则在$(a,b)$上,恒有$\varphi(x)>\psi(x)$
	\end{alertblock}\pause 
	\begin{exampleblock}{{\bf 例2}\hfill}
		证明:当$x>0$时,恒有
		$$x-\df 16x^3<\sin x<x.$$
	\end{exampleblock}
\end{frame}

\begin{frame}
	\linespread{1.2}
	\begin{alertblock}{{\bf 推论2}\hfill}
		设$\varphi(x),\psi(x) $均在$[a,b]$上{\b $n$阶可导},且:
		\begin{enumerate}
		  \item {\b $\varphi^{(n)}(x)>\psi^{(n)}(x),\;x\in(a,b)$}
		  \item {\b $\varphi^{(k)}(a)=\psi^{(k)}(a),k=0,1,2,\ldots,n-1$}
		\end{enumerate}
		则在$(a,b)$上,恒有$\varphi(x)>\psi(x)$
	\end{alertblock}
	\pause
	\begin{exampleblock}{{\bf 例3:}证明下列不等式\hfill}
% 		证明:当$x>0$时,
% 		$$\ln(1+x)>\df{\arctan x}{1+x}$$
		\begin{enumerate}
		  \item $\ln(1+x)>\df{\arctan x}{1+x},\;(x>0)$\pause
		  \item $e^x>1+x+\df
		  {x^2}{2!}+\df{x^3}{3!}+\ldots+\df{x^n}{n!},\;(x>0,n\in\mathbb{N})$
		\end{enumerate}
	\end{exampleblock}
\end{frame}

\section{可导函数的极值}

% \begin{frame}{函数单调性的判定}
% 	\linespread{1.2}\pause 
% 	\begin{block}{{\bf 定理5.4.1}\hfill P263}
% 		设$f(x)$在$[a,b]$上连续,$(a,b)$内可导,则\pause 
% 		\begin{enumerate}
% 		  \item 若对任意$x\in(a,b)$,$f\,'(x)>0$,则$f(x)$在$[a,b]$上严格单调递增\pause 
% 		  \item 若对任意$x\in(a,b)$,$f\,'(x)<0$,则$f(x)$在$[a,b]$上严格单调递减
% 		\end{enumerate}
% 	\end{block}\pause 
% 	\begin{exampleblock}{{\bf 例1}\hfill P264:例1-2}
% 		讨论$y=x-\sin x$与$y=e^x-x-1$的单调性。
% 	\end{exampleblock}
% \end{frame}

\begin{frame}{函数极值的判定}
	\linespread{1.2}\pause 
	\begin{block}{{\bf 定理5.4.2}(极值第一充分条件)\hfill P265}\pause 
		设$f(x)$在$x_0$连续,在$x_0$的去心领域内可导,且$f\,'(x)$在$x_0$两侧导数值异号,则
		$f(x)$在$x_0$处取极值。\pause 
	\end{block}
	\begin{exampleblock}{{\bf 例4:}讨论以下函数的极值\hfill }
		\begin{enumerate}
		  \item $f(x)=2-|x^3-1|$\pause
		  \item $f(x)=\left(1+x+\df{x^2}{2!}+\ldots+\df{x^n}{n!}\right)e^{-x}$ 
		\end{enumerate}
	\end{exampleblock}
\end{frame}

\begin{frame}{{函数极值的判定}{\small (续)}}
	\linespread{1.2}\pause 
	\begin{block}{{\bf 定理5.4.3}(极值第二充分条件)\hfill P267}\pause 
		设$f(x)$在$x_0$二阶可导,且$f\,'(x_0)=0$,则
		\begin{enumerate}\pause 
		  \item 若$f\,''(x_0)<0$,$f(x)$在$x_0$处取极大值\pause 
		  \item 若$f\,''(x_0)>0$,$f(x)$在$x_0$处取极小值
		\end{enumerate}
	\end{block}\pause 
	\begin{exampleblock}{{\bf 例5:}讨论以下函数的极值}
% 		求函数$f(x)=x^3-6x^2+5$的极值。
		\begin{enumerate}
		  \item $f(x)=x^3-6x^2+5$\hfill (P267-例4)\pause
		  \item $f(x)=\cos x+\df12\cos 2x$
		\end{enumerate}
	\end{exampleblock}
\end{frame}

% \begin{frame}{title}
% 	\linespread{1.2}
% 	\begin{block}{{\bf title}\hfill}
% 		123
% 	\end{block}
% \end{frame}

% \begin{frame}{综合应用}
% 	\linespread{1.2}\pause 
% 	\begin{exampleblock}{{\bf 例4}\hfill P268:例5}
% 		证明:当$0<x_1<x_2<\df{\pi}{2}$时,
% 		$$\df{\tan x_2}{\tan x_1}>\df{x_2}{x_1}$$
% 	\end{exampleblock}\pause 
% 	\begin{exampleblock}{{\bf 例5}\hfill P269:例6}
% 		证明:
% 		$$e^{-x}+\sin x<1+\df{x^2}{2},\;0<x<\df{\pi}{2}$$
% 	\end{exampleblock}
% \end{frame}

\section{综合应用}

\begin{frame}
	\linespread{1.2}
	\begin{exampleblock}{{\bf 例6}\hfill}
		函数$f(x)$对满足
		$$xf\,''(x)+3x[f\,'(x)]^2=1-e^{-x}$$
		\vspace{-2em}
		\begin{enumerate}
		  \item 若$f(x)$在$x=c\ne 0$处有极值,证明其必为极小值;\pause
		  \item 若$f(x)$在$x=0$处有极值,该极值为极大还是极小?
		\end{enumerate}
	\end{exampleblock}
	\pause
	\bigskip
		\begin{exampleblock}{{\bf 例7}\hfill P280:习题7}
		求数列$\sqrt[n]n$中最大的一项。
	\end{exampleblock}
\end{frame}

\begin{frame}
	\linespread{1.2}
	\begin{exampleblock}{{\bf 例8}\hfill P270:例7}
		判定方程$2^x-2x=1$有几个实根。
	\end{exampleblock}
	\pause
	\bigskip
	\begin{exampleblock}{{\bf 例9}\hfill}
		已知$x>0$时,
		$$kx+\df1{x^2}=1$$
		有且仅有一个根,求$k$的取值范围。
	\end{exampleblock}
	\pause
	{\bf 答案:}$k<0$或$k=\df29\sqrt 3$
\end{frame}

\begin{frame}
	\linespread{1.2}
	\begin{exampleblock}{{\bf 例10}\hfill}
		设$f(x)\in C[a,b]$,在$(a,b)$内$f\,''(x)>0$,试讨论
		$$F(x)=\df{f(x)-f(a)}{x-a}$$
		在$(a,b)$内的单调性。
	\end{exampleblock}
\end{frame}

% \begin{frame}
% 	\linespread{1.2}
% 	\begin{exampleblock}{{\bf 例8}\hfill P270:例9}
% 		垂直于宽度为$a$的河道,开挖一条宽度为$b$的运河,问能驶进运河的船只,最大长度是多少?
% 	\end{exampleblock}
% \end{frame}

% \section{函数的凹凸性}
% 
% \begin{frame}{函数的凹凸性}
% 	\linespread{1.2}\pause 
% 	{\bf 问题:}如何定义函数的凸、凹?\pause 
% 	\begin{block}{{\bf 定义5.4.1}(凹函数)\hfill P273}
% 		设函数$f(x)$在区间$I$上有定义,若对任意$x_1,x_2\in I$,以及任意$\lambda\in[0,1]$,
% 		有
% 		$$f[\lambda x_1+(1-\lambda)x_2]\leq \lambda f(x_1)+(1-\lambda)f(x_2) $$
% 		则称$f(x)$是区间$I$上的{\bb 凹函数}。\pause 若上式中的等号严格成立,则称其为{\bb 严格凹函数}。
% 	\end{block}
% \end{frame}
% 
% \begin{frame}{凹函数的充要条件}
% 	\linespread{1.2}\pause 
% 	\begin{block}{{\bf 定理5.4.4}\hfill P275}
% 		设$f(x)$在$(a,b)$内可导,则$f(x)$为$(a,b)$内的凹函数,当且仅当:对任意$x_1,x_2\in(a,b)$,
% 		恒有
% 		$$f(x_2)\geq f(x_1)+f\,'(x_1)(x_2-x_1). $$
% 		\pause 不等号严格成立时,对应于严格凹函数的情形。
% 	\end{block}
% \end{frame}
% 
% \begin{frame}{凹函数的判定}
% 	\linespread{1.2}\pause 
% 	\begin{block}{{\bf 定理5.4.5}\hfill P276}
% 		设$f(x)$在$(a,b)$二阶可导,则\pause 
% 		\begin{enumerate}
% 		  \item 若$f\,''(x)$恒不小于零,$f(x)$为凹函数\pause 
% 		  \item 若$f\,''(x)$恒不大于零,$f(x)$为凸函数
% 		\end{enumerate}
% 	\end{block}\pause 
% 	{\bb 拐点:}$f\,''(x)=0$的点\pause 
% 	\begin{block}{{\bf 推论}\hfill P276:例11}
% 		证明:严格凸(凹)函数的驻点为极值点。
% 	\end{block}
% \end{frame}
% 
% \section{分析作图法}
% 
% \begin{frame}{分析作图法}
% 	\linespread{1.2}\pause 
% 	\begin{exampleblock}{{\bf 例9}\hfill P279-例14}
% 		作出函数$y=\df{x}{1+x^2}$的图形。
% 	\end{exampleblock}\pause 
% 	\begin{enumerate}
% 	  \item {\bf 分析函数一般性质:}定义域、值域、奇偶性、周期性、与坐标轴的交点\pause 
% 	  \item {\bf 求一、二阶导函数:}确定不可导点\pause 
% 	  \item {\bf 列表分析:}单调、凸凹区间,极值点和拐点\pause 
% 	  \item {\bf 画出渐进线:}水平、铅直和斜渐进线\pause 
% 	  \item {\bf 描点作图}
% 	\end{enumerate}
% \end{frame}
% 
% \begin{frame}{渐进线}
% 	\linespread{1.2}\pause 
% 	\begin{block}{{\bf 定理3.3.5}\hfill P134}
% 		\begin{enumerate}\pause 
% 		  \item {\bb 水平渐近线:}$\limx{\pm\infty}f(x)=b$\pause 
% 		  \item {\bb 铅直渐近线:}$\limx{a^{\pm}}f(x)=\infty$\pause 
% 		  \item {\bb 斜渐近线:}\pause 
% 		  \begin{itemize}
% 		    \item 斜率:$k=\limx{\pm\infty}\df{f(x)}{x}$\pause 
% 		    \item 截距:$b=\limx{\pm\infty}[f(x)-kx]$
% 		  \end{itemize}
% 		\end{enumerate}
% 	\end{block}
% \end{frame}

\begin{frame}[<+->]{小结}
	\linespread{1.5}
%  	{\ba{问题:}}{\b 如何求闭区间上函数的最值和最值点?}\bigskip\pause
	\begin{enumerate}
	  \item {\bf 可导函数的单调性}
	  \begin{itemize}
	    \item 充分条件与充要条件
	    \item 用单调性证明不等式
% 	    \item 凸凹性
% 	    \item 拐点
	  \end{itemize}
	  \item {\bf 可导函数的极值}
% 	  \begin{itemize}
% 	    \item 充分条件与充要条件
% 	    \item 用单调性证明不等式
% % 	    \item 凸凹性
% % 	    \item 拐点
% 	  \end{itemize}
	\end{enumerate}
% 	\bigskip
% 	\pause
% 	\centerline{\ba{请自行阅读第六章\S 6.3节}}
\end{frame}

%=====================================

% \begin{frame}{title}
% 	\linespread{1.2}
% 	\begin{block}{{\bf title}\hfill}
% 		123
% 	\end{block}
% \end{frame}