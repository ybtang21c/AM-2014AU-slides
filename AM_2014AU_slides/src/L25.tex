\begin{frame}{第二十五讲、微积分基本公式与定积分的计算}
	\linespread{1.5}
	\begin{enumerate}
	  \item {\bf 内容与要求}{\b( \S 6.2,\S 6.3.3 )}
	  \begin{itemize}
	    \item 熟练掌握微积分基本公式
	    \item 理解变限积分函数及其性质
	    \item 熟练掌握定积分的计算方法
	  \vspace{1em}
	  \end{itemize}
	  \item {\bf 课后作业:}
	  \begin{itemize}
% 	    \item 书面作业:\\
	    \item {\b 习题6.2:1(5-8),2,3(3,4),4,6,7,8,13,15}
	    \item {\b 习题6.3:8,9,12,14,18}
%  	    \item 思考题:{\b 习题6.1:7,8,9,10}
	  \end{itemize}
	\end{enumerate}
\end{frame}

\begin{frame}[<+->]{复习与回顾}
	\linespread{1.2}
	\begin{enumerate}
	  \item {\bf 定积分的定义:}{\b 分割、取近似,做和、求极限}
	  $$\dint_a^bf(x)\d x=\lim\limits_{\lambda\to
		0}\sum\limits_{k=1}^nf(\xi_k)\Delta x_k$$
	  \item {\bf 定积分的性质}
	  \begin{itemize}
	    \item 线性性、区间可加性
	    \item 保号性、保序性、定积分的估值
	    \item 定积分中值定理
	    $$\dint_a^bf(x)\d x=f(\xi)(b-a)$$
	  \end{itemize}
	\end{enumerate}
\end{frame}

\section{微积分基本公式}

\begin{frame}{微积分基本公式}
	\linespread{1.2}\pause 
	\begin{block}{{\bf 定理6.2.1}(Newton-Leibnitz公式)\hfill P319}
		设函数$f(x)$在区间$[a,b]$上可积,$F(x)$是$f(x)$在$[a,b]$上的一个原函数,则
		$$\dint_a^bf(x)\d x=F(b)-F(a).$$
	\end{block}\pause 
	\begin{exampleblock}{{\bf 例1:}计算下列定积分\hfill P320}
		\begin{columns}
			\column{.30\textwidth}
			\begin{enumerate}\pause 
			  \item $\dint_{0}^{1}x^2 \d x$
			\end{enumerate}
			\column{.37\textwidth}
			\begin{enumerate}\pause 
			  \addtocounter{enumi}{1}
			  \item $\dint_{-1}^{\sqrt 3}\df 1{1+x^2}\d x$
			\end{enumerate}
			\column{.33\textwidth}
			\begin{enumerate}\pause 
			  \addtocounter{enumi}{2}
			  \item $\dint_{-1}^{-2}\df 1x\d x$
			\end{enumerate}
		\end{columns}
	\end{exampleblock}
\end{frame}

\section{变限积分及其性质}

\begin{frame}{变限积分的概念}
	\linespread{1.2}\pause 
	\begin{exampleblock}{{\bf 例2}\hfill }
		求$F(x)=\dint_0^x[\,t\,]\d t,\;(x>0)$的表达式,其中$[\,x\,]$为下取整函数。
	\end{exampleblock}\pause 
	\begin{block}{{\bf 定理6.2.2}\hfill P323}
		若$f(x)$在$[a,b]$上可积,则{\bb 变上限积分}
		$$\Phi(x)=\dint_a^xf(t)\d t$$
		在$[a,b]$上连续。
	\end{block}
% 	\begin{exampleblock}{{\bf 例2}\hfill P325-例8}
% 		已知$f(x)=\left\{\begin{array}{ll}
% 		x,&\quad 0\leq x\leq 1\\ 1, &\quad x>1
% 		\end{array}\right.$,求$F_1(x)=\dint_0^xf(t)dt$和$F_2(x)=\dint_1^xf(t)dt$
% 	\end{exampleblock}
\end{frame}

\begin{frame}{变限积分的性质}
	\linespread{1.2}\pause 
% 	\begin{block}{{\bf 定理6.2.2}\hfill P323}
% 		若$f(x)$在$[a,b]$上可积,则{\bb 变上限积分}
% 		$$\Phi(x)=\dint_a^xf(t)dt$$
% 		在$[a,b]$上连续。
% 	\end{block}
	\begin{block}{{\bf 定理6.2.3}\hfill P323}
		若$f(x)$在$[a,b]$上连续,则$\Phi(x)=\dint_a^xf(t)\d t$在$[a,b]$可导,且
		$$\Phi'(x)=f(x)$$
	\end{block}\pause 
	\begin{itemize}
	  \item \alert{若$f(x)$连续,则变上限积分$\Phi(x)=\dint_a^xf(t)
	  \d t$是$f(x)$的一个原函数}
	\end{itemize}
% 	{\bf 思考:}为什么$\Phi(x)$是$f(x)$的“一个”原函数?
\end{frame}

\begin{frame}{变限积分求导}
	\linespread{1.2}\pause 
	$$\alert{\left[\dint_{\varphi(x)}^{\psi(x)}f(t)\d t\right]'_x
	=f[\psi(x)]\psi'(x)-f[\varphi(x)]\varphi'(x)}$$\pause 
	\begin{exampleblock}{{\bf 例2:}求下列函数的导函数\hfill P325-例6}
		\begin{columns}[t]
			\column{.5\textwidth}
				\begin{enumerate}\pause 
				  \item $y=\dint_0^{x}e^{-t^2}\d t$\pause 
				  \item $y=\dint_0^{x^2}\sqrt{1+t^4}\d t$\pause 
				\end{enumerate}
			\column{.5\textwidth}
				\begin{enumerate}
				  \addtocounter{enumi}{2}
				  \item $y=\dint_{-x}^{\sqrt x}\sin t^2\d t$
				\end{enumerate}
		\end{columns}
	\end{exampleblock}
\end{frame}

\begin{frame}
	\linespread{1.2}
	\begin{exampleblock}{{\bf 例3}\hfill P329-例13}
		设$f(x)$在$[0,+\infty)$内连续,且$f(x)>0$,证明:
		$$F(x)=\df{\dint_0^xtf(t)\d t}{\dint_0^xf(t)\d t}$$
		在$(0,+\infty)$内单调递增。
	\end{exampleblock}
\end{frame}

\begin{frame}
	\linespread{1.2}
	\begin{exampleblock}{{\bf 例4}\hfill P330-例14}
		设$f(x)$在$[0,+\infty)$上可导,$f(0)=0$,且存在反函数$g(x)$,已知
		$$\dint_0^{f(x)}g(t)\d t=(x-1)e^x+x^2+1,$$
		求$f(x)$。
	\end{exampleblock}
\end{frame}

\section{定积分的计算}

\begin{frame}{定积分换元法与分部积分法}
	\linespread{1.2}\pause 
	\begin{block}{{\bf 定理6.3.3}(换元法)\hfill P346}
		设$f(x)$在$[a,b]$上连续,函数$x=\varphi(t)$满足:
		\begin{enumerate}
		  \item $\varphi(\alpha)=a,\;\varphi(\beta)=b$;
		  \item $\varphi(t)$在$[\alpha,\beta]$上连续可导,且$\varphi(t)\in[a,b]$,则
		\end{enumerate}
		$$\alert{\dint_a^bf(x)\d
		x=\dint_{\alpha}^{\beta}f[\varphi(t)]\varphi'(t)\d t}$$ \end{block}\pause 
	\begin{block}{{\bf 定理6.3.4}(分部积分法)\hfill P350}
		$$\alert{\dint_a^bu(x)\d v(x)=u(x)v(x)|_a^b-\dint_a^bv(x)\d u(x)}$$
	\end{block}
\end{frame}

\begin{frame}
	\linespread{1.2}
	\begin{exampleblock}{{\bf 例4:}计算下列定积分\hfill P347-例24-25}\pause 
% 		\begin{columns}
% 			\column{.5\textwidth}
				\begin{enumerate}
				  \item $\dint_9^4\df 1{\sqrt x}\d x$\pause 
				  \item $\dint_0^1\sqrt{(1-x^2)^3}\d x$\pause 
				  \item $\dint_0^{\ln 2}\sqrt{1-e^{-2x}}\d x$
				\end{enumerate}
% 			\column{.5\textwidth}
% 				\begin{enumerate}
% 				  \addtocounter{enumi}{2}
% 				  \item $\dint_{-\pi/2}^{\pi/2}\df{\sin^2x}{1+e^{-x}}\d x$
% 				\end{enumerate}
% 		\end{columns}
	\end{exampleblock}
\end{frame}

\begin{frame}{定积分的特殊计算方法}
	\linespread{1.5}
	\begin{enumerate}\pause 
	  \item {\bf 对称区间上的定积分:}%\pause 例如:
% 	  $$\dint_{-\pi/2}^{\pi/2}\df{\sin^2x}{1+e^{-x}}\d x$$\pause 
% 	  \item {\bf 奇、偶函数的定积分}\pause 
% 	  \begin{itemize}
% 	    \item 奇函数:$\dint_{-a}^af(x)=0$\pause 
% 	    \item 偶函数:$\dint_{-a}^af(x)=2\dint_0^af(x)$
% 	  \end{itemize}
	\end{enumerate}
	\pause
	\begin{exampleblock}{{\bf 例:}计算}
		\begin{enumerate}
		  \item $\dint_{-\pi/2}^{\pi/2}\df{\sin^2x}{1+e^{-x}}\d x$
		  \item $\dint_{-1}^1\df{x+\cos x}{1+\sin^2x}\d x$
		  \item $\dint_0^{\pi}\df{\cos x}{\sqrt{a^2\sin^2x+b^2\cos^2x}}\d
		  x\;\;(a^2+b^2\ne 0)$
		\end{enumerate}
	\end{exampleblock}
\end{frame}

% \begin{frame}{定积分的特殊计算方法}
% 	\linespread{1.5}
% 	\begin{enumerate}\pause 
% 	  \item {\bf 对称区间上的定积分:}\pause 例如:
% 	  $$\dint_{-\pi/2}^{\pi/2}\df{\sin^2x}{1+e^{-x}}\d x$$\pause 
% 	  \item {\bf 奇、偶函数的定积分}\pause 
% 	  \begin{itemize}
% 	    \item 奇函数:$\dint_{-a}^af(x)=0$\pause 
% 	    \item 偶函数:$\dint_{-a}^af(x)=2\dint_0^af(x)$
% 	  \end{itemize}
% 	\end{enumerate}
% \end{frame}

\begin{frame}
	\linespread{1.5}
	\begin{enumerate}
	  \addtocounter{enumi}{1}\pause 
	  \item {\bf 周期函数的定积分}\pause 
	  $$\dint_a^{a+T}f(x)\d x=\dint_0^{T}f(x)\d x$$\pause 
	  \vspace{-1em}
	  \item {\bf 正弦、余弦的转换:}\pause $$\dint_0^{\pi/2}f(\sin
	  x)\d x=\dint_0^{\pi/2}f(\cos x)\d x$$
	\end{enumerate}
	\pause
	\begin{exampleblock}{{\bf 例:}计算}
		$$I=\dint_0^{\pi/2}\df{\cos x}{\sin x+\cos x}\d x$$
	\end{exampleblock}
\end{frame}

\begin{frame}[<+->]{小结}
	\linespread{1.5}
	\begin{enumerate}
	  \item {\bf Newton-Leibnitz公式}
	  $$\alert{\dint_a^bf(x)\d x=F(b)-F(a)}$$
	  \item {\bf 变限积分}
	  $$\alert{\left[\dint_a^xf(t)dt\right]'_x=f(x)}$$
	  \item {\bf 定积分的计算}
	  \begin{itemize}
	    \item 换元法、分部积分法
	    \item \alert{定积分的一些特殊性质}
	  \end{itemize}
	\end{enumerate}
\end{frame}

%====================================

% \begin{frame}{title}
% 	\linespread{1.2}
% 	\begin{block}{{\bf title}\hfill}
% 		123
% 	\end{block}
% \end{frame}