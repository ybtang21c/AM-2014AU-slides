\begin{frame}
	\frametitle{习题课、导数与微分的应用}
	\linespread{1.5}
	{\bf 内容回顾}
	\begin{enumerate}
	    \item 中值定理
	    \begin{itemize}
	      \item Rolle定理、Lagrange中值定理、Cauchy中值定理
	    \end{itemize}
	    \item L'Hospital法则
	    \item Taylor公式
% 	    \begin{itemize}
% 	      \item 四则运算、反函数、复合函数
% 	      \item 隐函数与参数方程
% 	      \item 高阶导数
% 	    \end{itemize}
% 	    \item 微分的概念与应用
	\end{enumerate}
\end{frame}

% \section{极限的计算}

% \section{中值定理}

% \begin{frame}{中值定理与Taylor公式相关证明题}
% 	\linespread{1.5}
% 	\begin{enumerate}
% 	  \item {\bf 函数性质}
% 	  \begin{itemize}
% 	    \item 单调性、值域、凸凹性
% 	  \end{itemize}
% 	  \item {\bf }
% 	\end{enumerate}
% 	\begin{exampleblock}{{\bf title}\hfill }
% 		123
% 	\end{exampleblock}
% \end{frame}

\begin{frame}{中值定理与函数性质}
	\linespread{1.2}\pause 
	\begin{exampleblock}{{\bf 例1}\hfill }
		设$\limx{+\infty}f(x)$和$\limx{+\infty}f\,'(x)$均存在,证明:
		$$\limx{+\infty}f\,'(x)=0.$$
	\end{exampleblock}\pause 
% 	{\bf 思考:}
% 	\begin{itemize}
% 	  \item $\limx{+\infty}f(x)$和$\limx{+\infty}f\,'(x)$中有一个存在,是否能够推出另一个也存在?
% 	  \item 若$\limx{+\infty}f\,'(x)=l\ne 0$,能够推出$\limx{+\infty}f(x)$的什么性质?
% 	\end{itemize}
	\begin{exampleblock}{{\bf 例2}\hfill}
		设$f(x)$在$U(x_0)$内连续,在$U^0(x_0)$内可导,且$\limx{x_0}f\,'(x)=l$,则
		$$f\,'(x_0)=l.$$
	\end{exampleblock}
% 	\bigskip
% 	\begin{exampleblock}{{\bf 例2}\hfill }
% 		$f(x)$在$(a,+\infty)$上可导,$\limx{+\infty}f\,'(x)=0$,则:
% 		$$\limx{+\infty}\df{f(x)}{x}=0.$$
% 	\end{exampleblock}
\end{frame}

\begin{frame}{等式证明}
	\linespread{1.5}\pause 
	\begin{exampleblock}{{\bf 例3}\hfill}
		设$f(x)$在$[a,b]$连续,在$(a,b)$可导,且$a\geq 0$,则存在
		$x_1,x_2,x_3\in(a,b)$,使得
		$$f\,'(x_1)=(b+a)\df{f\,'(x_2)}{2x_2}=(a^2+ab+b^2)\df{f\,'(x_3)}{3x_3^2}$$
	\end{exampleblock}\pause 
	{\bf 注:}\alert{在使用中值定理前对要证明的等式进行必要的“整理”}
\end{frame}

\begin{frame}{辅助函数的构造}
	\linespread{1.2}\pause 
	\begin{exampleblock}{{\bf 例4}\hfill }
		设$f(x),g(x)\in C[a,b]$,在$(a,b)$内可导,且$f(a)=f(b)=0$,证明:存在
		$\xi\in(a,b)$,使得:
		\vspace{-1ex}
		$$f\,'(\xi)+f(\xi)g'(\xi)=0.$$
	\end{exampleblock}\pause 
	\begin{itemize}
	  \item \alert{$y+\lambda y'=0$:\pause $F(x)=e^{\lambda x}y$}\pause 
	  \item \alert{$ny+xy'=0$:\pause $F(x)=x^ny$}\pause 
	  \item \alert{$f\,'(x)g(x)-f(x)g'(x)=0$:\pause $F(x)=\df{f(x)}{g(x)}$}
	\end{itemize}
% 	\begin{exampleblock}{{\bf 例4}\hfill }
% 		设$f(x),g(x)$在$[a,b]$上可导,且恒不为零,\vspace{-1ex}
% 		$$\df{f\,'(x)}{f(x)}=\df{g'(x)}{g(x)},$$
% % 		\vspace{-2ex}
% 		证明:存在常数$k\ne 0$,使得$f(x)=kg(x)$对任意$x\in[a,b]$成立。
% 	\end{exampleblock}
\end{frame}

% \begin{frame}%{特殊点的性质}
% 	\linespread{1.5}
% 	\begin{exampleblock}{{\bf 例5}\hfill }
% 		证明:当$x\geq 0$时,
% 		$$\sqrt{x+1}-\sqrt{x}=\df{1}{2\sqrt{x+\theta_x}},$$
% 		其中$\df 14\leq \theta_x \leq \df 12$,且
% 		$$\limx{0^+}\theta_x=\df 14,\quad \limx{+\infty}\theta_x=\df 12.$$
% 	\end{exampleblock}
% \end{frame}

% \begin{frame}{辅助函数的构造}
% 	\linespread{1.2}
% 	\begin{exampleblock}{{\bf 例3}\hfill }
% 		设$f(x),g(x)\in C[a,b]$,在$(a,b)$内可导,且$f(a)=f(b)=0$,证明:存在
% 		$\xi\in(a,b)$,使得:
% 		$$f\,'(\xi)+f(\xi)g'(\xi)=0.$$
% 	\end{exampleblock}
% 	\begin{exampleblock}{{\bf 例4}\hfill }
% 		设$f(x),g(x)$在$[a,b]$上可导,且恒不为零,
% 		$$\df{f\,'(x)}{f(x)}=\df{g'(x)}{g(x)},$$
% 		证明:存在常数$k\ne 0$,使得$f(x)=kg(x)$对任意$x\in[a,b]$成立。
% 	\end{exampleblock}
% \end{frame}

\begin{frame}{证明不等式}
	\linespread{1.2}\pause 
	\begin{exampleblock}{{\bf 例6}\hfill}
		证明:若在$(a,b)$内,$f\,''(x)>0$,则对任意$a<x_1<x_2<b$,恒有
		$$f\left(\df{x_1+x_2}{2}\right)<\df{f(x_1)+f(x_2)}{2}.$$
	\end{exampleblock}\pause 
	{\bf 思考:}以上结论能否进一步推广?\pause 
	$$\alert{f(\lambda x_1+(1-\lambda)x_2)<\lambda
	f(x_1)+(1-\lambda)f(x_2),\;0<\lambda<1}$$
\end{frame}

\begin{frame}
	\linespread{1.5}
	\begin{exampleblock}{{\bf 例7}\hfill }
		$f(x)$在$[0,1]$上有二阶连续导数,且$f(0)=f(1)$,\\$|f\,''(x)|\leq A$,
		证明:对任意$x\in[0,1]$,恒有
		$$|f\,'(x)|\leq\df A2.$$
	\end{exampleblock}\pause 
	{\bf 用Taylor公式证明等式或不等式}\pause 
	\begin{itemize}
	  \item \alert{多使用Lagrange余项}\pause 
	  \item \alert{特殊的点:\pause 区间端点、\pause 极(最)值点、\pause
	  区间中点、\pause 距离为常数的点、\pause 已知条件中提到的特殊点}
	\end{itemize}
\end{frame}

\begin{frame}
	\linespread{1.5}
	\begin{exampleblock}{{\bf 例8}\hfill }
		设$f(x)$在$[a,b]$上二阶连续可导,且$f\,'(a)=f\,'(b)$\\ $=0$,
		则至少存在一点
		$c\in(a,b)$,使得
		$$|f\,''(c)|\geq \df 4{(b-a)^2}|f(b)-f(a)|.$$
	\end{exampleblock}
\end{frame}

\begin{frame}{极限的计算}
	\linespread{1.2}
	\begin{exampleblock}{{\bf 例9:}计算下列极限\hfill}
		\begin{columns}\pause 
			\column{.48\textwidth}
				\begin{enumerate}
				  \item $\limx{0}\df{e^x-e^{-x}-2x}{x-\sin x}$\pause 
				  \item $\limx{0}\df{e^{-1/x^2}}{x^{100}}$\pause 
				  \item $\limx{a}\df{x^a-a^x}{x^x-a^a}\;(a>0)$\pause 
				\end{enumerate}	
			\column{.52\textwidth}
				\begin{enumerate}
				  \addtocounter{enumi}{3}
				  \item $\limx{0}\left(\df{a^x+b^x+c^x}{3}\right)^{1/x}$\pause 
				  \item $\limn\sqrt{n}(\sqrt[n]{n}-1)$\pause 
				  \item $\limn\left(n\sin\df{1}{n}\right)^{n^2}$
				\end{enumerate}	
		\end{columns}
	\end{exampleblock}
\end{frame}

\begin{frame}{课堂练习}
	\linespread{2}
	\begin{exampleblock}{{\bf 例10:}计算下列极限\hfill}
		\begin{enumerate}
% 		  \item $\limx{0}\df{e^x-1-x}{\df{1}{\sqrt{1-x}-\cos\sqrt x}}$
		  \item $\limx{0}\df{\cos(\sin x)-\cos x}{x^4}$
		  \item $\limx{+\infty}\left[x-x^2\ln\left(1+\df 1x\right)\right]$
		\end{enumerate}
	\end{exampleblock}
\end{frame}



% \begin{frame}{函数性质}
% 	\linespread{1.2}
% 	\begin{exampleblock}{{\bf 例15}\hfill}
% 		证明:若在$(a,b)$内,$f\,''(x)>0$,则对任意$a<x_1<x_2<b$,恒有
% 		$$f\left(\df{x_1+x_2}{2}\right)<\df{f(x_1)+f(x_2)}{2}.$$
% 	\end{exampleblock}\pause 
% 	{\bf 思考:}以上结论能否进一步推广?\pause 
% 	$$\alert{f(\lambda x_1+(1-\lambda)x_2)<\lambda
% 	f(x_1)+(1-\lambda)f(x_2),\;0<\lambda<1}$$
% \end{frame}
% 
% % \section{辅助函数的构造}
% 
% 
% 

%=================================

% \begin{frame}{title}
% 	\linespread{1.5}
% 	\begin{exampleblock}{{\bf title}\hfill }
% 		123
% 	\end{exampleblock}
% \end{frame}