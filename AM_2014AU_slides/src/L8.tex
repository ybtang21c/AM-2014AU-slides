\begin{frame}
	\frametitle{第八讲、函数极限存在的判定准则}
	\linespread{1.5}
	\begin{enumerate}
	  \item {\bf 内容与要求}{\color{blue}( \S3.2 )}
	  \begin{itemize}
	    \item 掌握复合函数极限的性质
	    \item 理解函数极限与数列极限的关系
	    \item 熟练掌握两个常用极限及其应用
	  \vspace{1em}
	  \end{itemize}
	  \item {\bf 课后练习:}
	  \begin{itemize}
	    \item 书面作业:{\b 习题3.2:2,3,5,6}
	    \item 思考题:{\b 习题3.2:4,7}
	  \end{itemize}
	\end{enumerate}
\end{frame}

\begin{frame}[<+->]{复习与回顾}
	\linespread{1.5}
	\begin{enumerate}
	  \item {\bf 六种不同的趋势:}
	  \begin{itemize}
	    \item \alert{$x\to\infty,\,+\infty,\,-\infty,\,x_0,\,x_0^+,\,x_0^-$}
	  \end{itemize}
	  \item {\bf 极限的“$\e-\delta$”定义}
	  \begin{itemize}
	    \item \alert{$\limx{x_0}f(x)=A:\forall\e>0,\exists\delta>0,\forall
	    0<|x-x_0|<\delta,|f(x)-A|<e$}
	  \end{itemize}
	  \item {\bf 函数极限的基本性质}
	  \begin{itemize}
	    \item 唯一性、有界性、保号性、四则运算
	  \end{itemize}
	\end{enumerate}
\end{frame}

\section{复合函数的极限}

\begin{frame}{复合函数的极限}
	\linespread{1.2}\pause 
	\begin{block}{{\bf 定理3.2.2}\hfill P116}
		设有复合函数$y=f[g(x)]$,其中
		$$\limx{x_0}g(x)=u_0,\quad\lim\limits_{u\to u_0}f(u)=A,$$
		在$x_0$附近$g(x)\ne u_0$,则
		$$\limx{x_0}f[g(x)]=A.$$
	\end{block}\pause 
	{\bf 注:}以上结论可以推广到$x$趋于无穷的情形
\end{frame}

\section{函数极限与数列极限的关系}

\begin{frame}{函数极限与数列极限的关系}
	\linespread{1.2}\pause 
	\begin{block}{{\bf 定理3.2.3}(Hiene定理)\hfill P117}
		$\limx{\Delta}f(x)=A\Leftrightarrow
		$若数列$\{x_n\}$满足:$x_n\to \Delta(n\to$ $\infty)$,则
		$$\limn f(x_n)=A$$
	\end{block}\pause 
	\begin{itemize}
	  \item 以上$\Delta$对应于函数极限的六种不同趋势\pause 
	  \item {\bf 用途一:}证明极限不存在性\pause
	  \item {\bf 用途二:}利用函数极限计算对应的数列极限
	\end{itemize}
\end{frame}

\begin{frame}
	\linespread{1.2}
	\begin{exampleblock}{{\bf 例1}\hfill P118}
		证明:$f(x)=\sin\df 1x$当$x\to 0$时无极限。
	\end{exampleblock}\pause 
	\bigskip
	\begin{exampleblock}{{\bf 例2}\hfill 习题3.2-7}
		证明Dirichlet函数在任意点处无极限。
	\end{exampleblock}
\end{frame}

\section{函数极限的夹逼定理}

\begin{frame}{函数极限的夹逼定理}
	\linespread{1.2}\pause 
	\begin{block}{{\bf 定理3.2.5}\hfill P119}
		设在$x_0$的某领域内,恒有
		$$\varphi(x)\leq f(x)\leq\psi(x), $$
		且$\limx{x_0}\varphi(x)=\limx{x_0}\psi(x)=A$,则
		$$\limx{x_0}f(x)=A.$$
	\end{block}
\end{frame}

% \subsection{重要极限一:$\limx{\infty}\left(1+\df 1x\right)^x=e$}

\begin{frame}{重要极限一}
	\linespread{1.2}
	\begin{alertblock}{{\bf 例3}\hfill P120-例9}
		证明:$\limx{\infty}\left(1+\df 1x\right)^x=e$
	\end{alertblock}\pause 
	\begin{exampleblock}{{\bf 例4:}计算下列极限\hfill}\pause 
		\begin{columns}
			\column{.45\textwidth}
				\begin{enumerate}
				  \item $\limx{0}(1+x)^{1/x}$\pause 
				  \item $\limx 0(1+\sin x)^{1/\sin x}$\pause 
				  \item $\limx 0\df{\ln(1+ax)}{x}$\pause 
				\end{enumerate}
			\column{.55\textwidth}
				\begin{enumerate}
				  \addtocounter{enumi}{3}
				  \item $\limx 0\df{e^{ax}-1}{x}$\pause 
				  \item $\limx 0\df{a^x-1}{x}(a>0)$\pause 
				  \item $\limn n(\sqrt[n]{a}-1) (a>0)$
				\end{enumerate}
		\end{columns}
	\end{exampleblock}
\end{frame}

% \subsection{重要极限二:$\limx{0}\df {\sin x}x=1$}

\begin{frame}{重要极限二}
	\linespread{1.2}
	\begin{alertblock}{{\bf 例5}\hfill P121-例11}
		证明:$\limx{0}\df {\sin x}x=1$
	\end{alertblock}\pause 
	\begin{exampleblock}{{\bf 例6:}计算下列极限\hfill}\pause 
		\begin{columns}[t]
			\column{.5\textwidth}
				\begin{enumerate}
				  \item $\limx{0}\df{\sin\sin x}{\sin x}$\pause 
				  \item $\limx 0\df{1-\cos x}{x^2}$\pause 
				  \item $\limx 0\df{\sin mx}{\sin nx}(n\ne 0)$\pause 
				\end{enumerate}
			\column{.5\textwidth}
				\begin{enumerate}
				  \addtocounter{enumi}{3}
				  \item $\limx 0\df{\tan x}{x}$\pause 
				  \item $\limx a\df{\sin x-\sin a}{x-a}$
% 				  \item $\limn n(\sqrt[n]{a}-1) (a>0)$
				\end{enumerate}
		\end{columns}
	\end{exampleblock}
\end{frame}

\begin{frame}[<+->]{小结}
	\linespread{1.5}
	\begin{enumerate}
	  \item {\bf 函数极限存在性的判定:}
	  \begin{itemize}
	    \item 复合函数
	    \item 函数极限与数列极限(Hiene定理)
	    \item 夹逼定理
	  \end{itemize}
	  \item {\bf 两个重要极限及其应用}
	  \begin{itemize}
	    \item $\limx{\infty}\left(1+\df 1x\right)^x=e$
	    \item $\limx{0}\df {\sin x}x=1$
	  \end{itemize}
	\end{enumerate}
\end{frame}