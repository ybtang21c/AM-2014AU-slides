\begin{frame}
	\frametitle{第十一讲、导数}
	\linespread{1.5}
	\begin{enumerate}
	  \item {\bf 内容与要求}{\color{blue}( \S4.1 )}
	  \begin{itemize}
	    \item 理解导数的定义
	    \item 熟练掌握基本初等函数的导函数
	    \item 理解导数存在的条件
	  \vspace{1em}
	  \end{itemize}
	  \item {\bf 课后练习:}
	  \begin{itemize}
	    \item 书面作业:{\b 习题4.1:6,7,8,9,10}
	    \item 思考题:{\b 习题4.1:12,13,14,15,16}
	  \end{itemize}
	\end{enumerate}
\end{frame}

\section{导数的概念}

\begin{frame}{函数在一点处的导数}
	\linespread{1.2}\pause 
	\begin{block}{{\bf 定义4.1.1} \hfill P156}
		函数$y=f(x)$在$x_0$的某领域内有定义,\pause 若
		$$\limdx\df{f(x_0+\dx)-f(x_0)}{\dx}$$
		存在,\pause 则称其为{\bb $f(x)$在$x_0$处的导数},\pause 记为
		$$f\,'(x_0),\quad \left.\df{\d y}{\d x}\right|_{x=x_0},\quad y'_x|_{x=x_0}$$
	\end{block}
\end{frame}

\begin{frame}
	\linespread{2}
	\begin{exampleblock}{{\bf 例1} \hfill 习题4.1-2}
		假设$f\,'(x_0)$存在,则
		\begin{enumerate}\pause 
		  \item $\limdx\df{f(x_0-\Delta x)-f(x_0)}{\Delta
		  x}=$\pause \underline{\quad\alert{$-f\,'(x_0)$}\quad}\pause 
		  \item $\lim\limits_{h\to 0}\df{f(x_0+2h)-f(x_0)}{h}=$
		  \pause \underline{\quad\alert{$2f\,'(x_0)$}\quad}\pause 
		  \item $\lim\limits_{h\to 0}\df{f(x_0+h)-f(x_0-h)}{h}=$\pause 
		  \underline{\quad\alert{$2f\,'(x_0)$}\quad}
		\end{enumerate}
	\end{exampleblock}\pause 
	{\ba{ 思考:}}若以上某个极限存在,是否就意味着$f(x)$在$x_0$可导?
\end{frame}

\begin{frame}{导数的物理/几何意义}
	\linespread{1.5}\pause 
	\begin{itemize}
	  \item {\bf 变速直线运动的瞬时速度}\pause
	  $$v(t_0)=\lim\limits_{t\to t_0}\df{S(t)-S(t_0)}{t-t_0}$$
	\end{itemize}
	\pause
% 	\begin{exampleblock}{{\bf 例1} \hfill }
% 		一辆汽车做直线运动,其位移$S(t)=1-t^4$,求其在$t=2$是的速度。
% 	\end{exampleblock}\pause
	\begin{itemize}
	  \item {\bf 曲线在一点处切线的斜率:}\pause
	  $$k(x_0)=\lim\limits_{x\to x_0}\df{f(x)-f(x_0)}{x-x_0}$$
	\end{itemize} 
% 	{\ba{以直代曲:}在可导点的附近,函数曲线与其切线近似相等}
	\pause
	\begin{itemize}
	  \item \alert{\bf 导数:函数关于自变量的相对变化率}
	\end{itemize}
\end{frame}

\begin{frame}{导数的几何意义}
	\linespread{1.5}\pause 
	\begin{exampleblock}{{\bf 例2} \hfill P156-例1}
		函数$f(x)=\sqrt[3]x$在点$x=0$是否可导?
	\end{exampleblock}\pause 
	{\ba{思考:}}可导等价于有切线吗?\pause (\alert{否})\pause 
	\bigskip
	\begin{exampleblock}{{\bf 例3} \hfill }
		已知函数$f(x)$在点$x_0$处可导,求曲线$y=f(x)$在该点的切线和法线方程。
	\end{exampleblock}
% 	{\ba{以直代曲:}在可导点的附近,函数曲线与其切线近似相等}
\end{frame}

\section{导函数}

% \begin{frame}{导函数}
% 	\linespread{1.2}
% 	\begin{block}{{\bf 定义4.1.1} \hfill}
% 		123
% 	\end{block}
% \end{frame}


\begin{frame}{导函数}
	\linespread{1.2}\pause 
	\begin{alertblock}{{\bf 例4}(一些常用函数的导函数) \hfill }\pause 
% 		求以下函数的导数:
		\begin{enumerate}
		  \item $f(x)=C\;(C\mbox{为常数})$\pause \hfill $f\,'(x)=0$\pause 
		  \item $f(x)=x^n\;(n\in\mathbb{Z})$\pause \hfill $f\,'(x)=nx^{n-1}\,(n\ne
		  0)$\pause 
		  \item $f(x)=e^x$\pause \hfill $f\,'(x)=e^x$\pause 
		  \item $f(x)=\ln x$\pause \hfill $f\,'(x)=\df 1x$\pause 
		  \item $f(x)=\sin x$\pause \hfill $f\,'(x)=\cos x$\pause 
		  \item $f(x)=\cos x$\pause \hfill $f\,'(x)=-\sin x$
		\end{enumerate}
	\end{alertblock}
\end{frame}

\section{导数存在的条件}

\begin{frame}{导数存在的条件}
	\linespread{1.5}\pause 
	\begin{block}{{\bf 定理4.1.1} \hfill P161}
		$f(x)$在$x_0$可导,当且仅当在该点的左、右导数存在且相等。
	\end{block}\pause 
	{\ba{思考:}}$f(x)=|{\sqrt[3]x}|$在原点是否可导?\pause
% 	{\ba{注:}}左、右侧切线重合不等于可导!
	\begin{block}{{\bf 定理4.1.2} \hfill P161}
		$f(x)$在一点可导,则一定在该点连续。
	\end{block}\pause 
	{\ba{以直代曲:}}可导点附近,函数曲线与其切线近似相等
\end{frame}

\begin{frame}
	\linespread{1.2}
	\begin{exampleblock}{{\bf 例5} \hfill P164-例8}
		确定常数$a,b$的值,使得函数
		$$f(x)=\left\{
		\begin{array}{ll}
		ax+b,& x>0\\
		e^x,& x\leq 0
		\end{array}
		\right.$$
		在$x=0$可导。
	\end{exampleblock}
\end{frame}

\begin{frame}
	\linespread{1.2}
	\begin{exampleblock}{{\bf 例6} \hfill P160-例5}
		证明:双曲线$y=\df 1x$的两个分支没有公共的切线。
	\end{exampleblock}
\end{frame}

\begin{frame}[<+->]{小结}
	\linespread{1.5}
	\begin{enumerate}
	  \item {\bf 导数的概念:}
	  \begin{itemize}
	    \item $f\,'(x_0)=\limdx\df{f(x_0+\dx)-f(x_0)}{\d x}$
	  \end{itemize}
	  \item {\bf 常用初等函数的导函数}
	  \begin{itemize}
	    \item $C,\;x^n,\;e^x,\;\ln x,\;\sin x,\;\cos x$
	  \end{itemize}
	  \item {\bf 函数可导的条件}
	  \begin{itemize}
	    \item 可导必连续
	    \item 以直代曲:$\Delta y=y'\Delta x+\circ(\Delta x),\,(\Delta x\to 0)$
	  \end{itemize}
	\end{enumerate}
\end{frame}

%======================

% \begin{frame}{123}
% 	\linespread{1.2}
% 	\begin{block}{{\bf 定义4.1.1} \hfill}
% 		123
% 	\end{block}
% \end{frame}
